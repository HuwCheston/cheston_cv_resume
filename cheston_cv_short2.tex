\documentclass[11pt]{res} % Use the res.cls style, the font size can be changed to 11pt or 12pt here
\usepackage{helvet} % Default font is the helvetica postscript font
%\usepackage{newcent} % To change the default font to the new century schoolbook postscript font uncomment this line and comment the one above
\usepackage[colorlinks=true, urlcolor=blue, linkcolor=red]{hyperref}
\usepackage{hyperref,textcomp,fontawesome5,fancyhdr,hanging}
\newsectionwidth{0pt} % Stops section indenting
\begin{document}
%----------------------------------------------------------------------------------------
%	NAME AND ADDRESS(ES)
%----------------------------------------------------------------------------------------
\name{HUW WILLIAM CHESTON}
%----------------------------------------------------------------------------------------
\begin{resume}
%------------------------------------------------------------==--------------
\vspace{8pt}
\centerline{\href{mailto:hwc31@cam.ac.uk}{\underline{Email}} \hspace{4pt} $\mid$ \hspace{4pt} \href{https://huwcheston.github.io/}{\underline{Webpage}} \hspace{4pt} $\mid$ \hspace{4pt} \href{https://www.github.com/huwcheston}{\underline{GitHub}} \hspace{4pt} $\mid$ \hspace{4pt} \href{https://scholar.google.com/citations?user=byas-BIAAAAJ&hl=en}{\underline{Google Scholar}} \hspace{4pt} $\mid$ \hspace{4pt} \href{https://twitter.com/huwcheston}{\underline{@HuwCheston}}}
%------------------------------------------------------------==--------------
%	EDUCATION 
%----------------------------------------------------------------------------
\section{\centerline{EDUCATION}} 
\vspace{8pt} % Gap between title and text
\textbf{Ph.D., Music Data Science}, University of Cambridge  \hfill \emph{expected} October, 2024   \\
\-\hspace{.25in} \textbf{Thesis:} New Methods in Computational Modeling of Musical Improvisation \\
\-\hspace{.25in} \emph{Applied techniques from machine learning and data science to analyzing audio recordings}

\textbf{MSt., Music Psychology}, University of Oxford (\emph{Distinction}) \hfill October, 2020\\
\-\hspace{.25in} Final average of 85\%, \textit{graduated with the highest mark in the year}

\textbf{BA., Music}, University of Oxford (\emph{First Class}) \hfill June, 2019 \\
\-\hspace{.25in} Final average of 76\%, \textit{graduated with the highest mark in the year}

%----------------------------------------------------------------------------------------
%	Experience
%----------------------------------------------------------------------------------------
\section{\centerline{EXPERIENCE}}
\vspace{8pt} 
\textbf{Research Scientist Intern}, Spotify \hfill June 2024 -- \\
Worked in close collaboration with members of the Audio Intelligence and Machine Learning teams to develop new approaches to audio and music classification.

\textbf{Music \& Science Module Leader}, Cambridge Summer School, Sutton Trust \hfill August 2023 \\
Designed \& delivered workshop on AI-assisted audio analysis for students (ages 16–17)

\textbf{Supervisor \& Guest Lecturer}, University of Cambridge \hfill January 2022 – May 2024
\begin{itemize}
\setlength\itemsep{0pt}
    \item[--]Delivered over 100 small-group supervisions for Undergraduate students (natural sciences \& humanities courses) on: (\emph{i}) analysing audio recordings, (\emph{ii}) introduction to programming in Python and R, (\emph{iii}) visualising and simulating data in Python and R. 
    \item[--]Delivered yearly lecture on analysing audio recordings for Undergraduate \emph{Music \& Science} course, involving introduction and demonstration of Music Information Retrieval concepts.
    \item[--]Supervised data-driven and analytical Undergraduate dissertation projects.
    \item[--]Used Jupyter Book to build \href{https://huwcheston.github.io/PS-Supervision/intro.html}{interactive web app} for hosting teaching materials
\end{itemize}

\textbf{Graduate Music Assistant}, Kingswood School, Bath, UK \hfill September 2020 -- June 2021 \\
Duties involved: (\emph{i}) planning \& teaching classes, including virtually during the pandemic; (\emph{ii}) managing school’s recording studio; (\emph{iii}) delivering extra-curricular music performance and audio production classes; (\emph{iv}) producing promotional audio-visual material for the school.

\textbf{Professional Musician \& Sound Technician} \href{https://www.youtube.com/watch?v=2ccWNviTf4A&ab_channel=HuwCheston}{[Showreel link]} \hfill September 2016 –

%----------------------------------------------------------------------------------------
%	PUBLICATIONS SECTION (APA)
%----------------------------------------------------------------------------------------
\section{\centerline{SELECTED PUBLICATIONS}}
\vspace{8pt} 
\begin{hangparas}{.25in}{1}
\textbf{Cheston, H.}, Cross, I., \& Harrison, P. M. C. (2024, in press). Trade-offs in Coordination Strategies for Duet Jazz Performances Subject to Network Delay and Jitter. \emph{Music Perception}.
\begin{itemize}
\setlength\itemsep{0pt}
    \item[--]Built software platform with OpenCV to record multiple audio-video streams in real-time
    \item[--]Used time series analysis to model and forecast data from multiple sources
    \item[--]Automated project documentation building and hosting with Sphinx and GitHub Pages
\end{itemize}

\textbf{Cheston, H.}, Schlichting, J. S., Cross, I., \& Harrison, P. M. C. (2024). Rhythmic Qualities of Jazz Improvisation Predict Performer Identity and Style in Source-Separated Audio Recordings. \emph{PsyArXiv}. \href{https://doi.org/10.31234/osf.io/txy2f}{[DOI: 10.31234/osf.io/txy2f]}
\begin{itemize}
\setlength\itemsep{0pt}
    \item[--]Built classification model for identifying performers featured on commercial audio recordings
    \item[--]Performed hierarchical clustering to classify performers into genre based on model output
    \item[--]Created \href{https://huwcheston.github.io/Cambridge-Jazz-Trio-Database/_static/prediction-app.html}{interactive web application} using jQuery and Plotly to visualise the model predictions
    \item[--]Hosted model online to enable users to process their own recordings
\end{itemize}

\textbf{Cheston, H.}, Schlichting, J. S., Cross, I., \& Harrison, P. M. C. (2024). Cambridge Jazz Trio Database: Automated Timing Annotation of Jazz Piano Trio Recordings Processed Using Audio Source Separation. \emph{PsyArXiv}. \href{https://doi.org/10.31234/osf.io/jyqp3}{[DOI: 10.31234/osf.io/jyqp3]}.
\begin{itemize}
\setlength\itemsep{0pt}
    \item[--]Developed an audio signal processing pipeline for extracting data from audio recordings
    \item[--]Optimized pipeline performance using nonlinear optimization algorithms
    \item[--]Developed \href{https://huwcheston.github.io/Cambridge-Jazz-Trio-Database/}{interactive web application} for exploring database using jQuery for UI
\end{itemize}
\end{hangparas}

%----------------------------------------------------------------------------------------
%	AWARDS AND PRIZES
%----------------------------------------------------------------------------------------
\section{\centerline{SELECTED AWARDS AND PRIZES}}
\vspace{8pt} % Gap between title and text
\textbf{Project Incubation Award} (\emph{£2000}), Cambridge Digital Humanities \hfill  May, 2022\\
Awarded for development and testing of \emph{Audio-Visual Manipulator} software \href{https://www.cdh.cam.ac.uk/research/cdh-funding/newmusicsoftwareplatform/}{[Project page]}

\textbf{Vice-Chancellor's Award} (\emph{£75,000}), Cambridge Trust  \hfill  September, 2021 \\
Full scholarship (fees \& stipend) for Ph.D study

\textbf{Music Prize} (\emph{£100}), University of Oxford  \hfill  October, 2020\\
Awarded for highest average mark in 2020 MSt. cohort

\textbf{Louis Curran Scholarship} (\emph{£25,000}), Linacre College, University of Oxford \hfill  August, 2019\\
Full scholarship (fees \& stipend) for MSt. study

\textbf{Gibbs Prize} (\emph{£500}), University of Oxford \hfill  June, 2019 \\
Awarded for highest average mark in 2019 BA. cohort

\textbf{Academic Scholarship} (\emph{£300} x2), Christ Church, University of Oxford \hfill  2017; 2018

%----------------------------------------------------------------------------------------
%	TECHNICAL SKILLS
%----------------------------------------------------------------------------------------
\section{\centerline{TECHNICAL SKILLS}}
\vspace{16pt} % Gap between title and text
\begin{itemize}
\setlength\itemsep{0pt}
    \item[--]\textbf{Languages}: Python \faPython, \textsf{R}, JavaScript, HTML/CSS 
    \item[--]\textbf{Developer Tools}: \faGit, \LaTeX, Docker, Google Cloud Platform, Excel, PyCharm, Jupyter
    \item[--]\textbf{Libraries}: pandas, NumPy, Matplotlib, Scikit-learn, SciPy, Statsmodels, OpenCV, pytest
\vspace{0pt}
\end{itemize}
%----------------------------------------------------------------------------------------
%	REFERENCES
%----------------------------------------------------------------------------------------
\section{\centerline{REFERENCES AVAILABLE ON REQUEST}}
\end{resume} 
\end{document}