\documentclass[11pt]{stylesheet} % Use the res.cls style, the font size can be changed to 11pt or 12pt here
\usepackage{helvet} % Default font is the helvetica postscript font
%\usepackage{newcent} % To change the default font to the new century schoolbook postscript font uncomment this line and comment the one above
\usepackage[colorlinks=true, urlcolor=blue, linkcolor=red]{hyperref}
\usepackage{hyperref,textcomp,fontawesome5,fancyhdr, hanging}
\usepackage{enumitem}
\newsectionwidth{0pt} % Stops section indenting
\begin{document}
%----------------------------------------------------------------------------------------
%	NAME AND ADDRESS(ES)
%----------------------------------------------------------------------------------------
\name{HUW WILLIAM CHESTON}\vspace{8pt}

%----------------------------------------------------------------------------------------
\begin{resume}
%----------------------------------------------------------------------------------------
\centerline{\faEnvelope\href{mailto:hwc31@cam.ac.uk}{\underline{Email}} \hspace{4pt} $\mid$ \hspace{4pt} \faGlobe\href{https://huwcheston.github.io/}{\underline{Webpage}} \hspace{4pt} $\mid$ \hspace{4pt} \faGithub\href{https://www.github.com/huwcheston}{\underline{GitHub}} \hspace{4pt} $\mid$ \hspace{4pt} \faBook\href{https://scholar.google.com/citations?user=byas-BIAAAAJ&hl=en}{\underline{Google Scholar}} \hspace{4pt}}

%----------------------------------------------------------------------------------------
%	EDUCATION 
%----------------------------------------------------------------------------------------
\section{\centerline{EDUCATION}}\label{sec:education}
\vspace{8pt} % Gap between title and text
\textbf{Ph.D., Music Information Retrieval}  \hfill \emph{expected} Summer, 2025   \\
Centre for Music \& Science, University of Cambridge \\
\-\hspace{1cm} \textbf{Thesis:} New Methods in Computational Modeling and of Improvised Music \\
\-\hspace{1cm} \emph{Supervisors:} Peter Harrison \& Ian Cross \\
\-\hspace{1cm} \emph{Generated interpretable ML models for identifying musicians from audio recordings.}

\textbf{MSt., Music Psychology}, \emph{Distinction}, 86\% average (highest mark in year) \hfill October, 2020\\
Linacre College, University of Oxford

\textbf{BA., Music}, \emph{First Class}, 76\% average (highest mark in year) \hfill June, 2019 \\ 
Christ Church, University of Oxford 
%----------------------------------------------------------------------------------------
%	EXPERIENCE
%----------------------------------------------------------------------------------------
\section{\centerline{EMPLOYMENT}}\label{sec:experience}\vspace{-8pt}
\subsection{RESEARCH}\vspace{-8pt}
\textbf{Research Scientist}, PhD Internship @ Spotify \href{https://research.atspotify.com/audio-intelligence/}{[Audio Intelligence lab]} \hfill Summer 2024
\begin{itemize}[leftmargin=*]
	\setlength\itemsep{0pt}
		\item[--]Developed end-to-end ML model for identifying samples (\emph{e.g.,} drum breaks) in audio recordings, facilitating downstream plagiarism detection \& music recommendation services.
		\item[--]Managed pipelines for generating artificial training data at petabyte-scale in Beam \& GCS.
		\item[--]Deployed PyTorch training runs on distributed GPU clusters using Ray \& Kubernetes
		\item[--]Final model outperformed baseline Shazam-style algorithm by 9x, setting a new state-of-the-art.
		\item[--]Presented results to major company stakeholders \& senior research managers.
		\item[--]Results presented in interactive web application and in a scientific publication.
\end{itemize}\vspace{-16pt}
\subsection{TEACHING}\vspace{-8pt}
\textbf{Module Leader}, Sutton Trust, Cambridge \hfill Summer 2023
\begin{itemize}[leftmargin=*]
	\setlength\itemsep{0pt}
		\item[--]Delivered workshops in music computing for students in final years of secondary state education. 
		\item[--]Interactive module materials hosted on GitHub Pages and Google Colab.
\end{itemize}
\textbf{Supervisor \& Lecturer}, University of Cambridge \hfill Winter 2021 --- Summer 2024
\begin{itemize}[leftmargin=*]
	\setlength\itemsep{0pt}
		\item[--]Delivered 100+ supervisions for sciences \& humanities undergraduates, covering (i): statistical programming in Python and \textsf{R}, (ii) visualising \& simulating data, (iii) writing research proposals.
		\item[--]Mentored undergraduate research projects in data science and machine learning topics.
		\item[--]Delivered yearly undergraduate lectures on computational modelling of audio data
		\item[--]Interactive teaching materials hosted online using \href{https://huwcheston.github.io/PS-Supervision/intro.html}{Jupyter Book}
\end{itemize}
\textbf{Graduate Teaching Assistant}, Kingswood School, Bath \hfill September 2020 --- June 2021
\begin{itemize}[leftmargin=*]
	\setlength\itemsep{0pt}
		\item[--]Planned \& taught music technology lessons, in-classroom and virtually during COVID lockdowns.
		\item[--]Managed recording studio and produced promotional audio-visual material for the school.
\end{itemize}\vspace{-16pt}
\subsection{OTHER}\vspace{-8pt}
\textbf{Musician \& Sound Engineer}, Freelance \hfill September 2016 --- \\
Performance (guitar, bass guitar), sound mixing, mastering work across the UK for clients including Blenheim Palace, TRUCK Festival, ATMOSPHERE Opera Festival, Oxford Playhouse. \href{https://www.youtube.com/watch?v=2ccWNviTf4A&ab_channel=HuwCheston}{[Showreel]} \hfill 

%----------------------------------------------------------------------------------------
%	PUBLICATIONS SECTION (APA)
%----------------------------------------------------------------------------------------
\section{\centerline{PUBLICATIONS}}\label{sec:publications}
\subsection{PEER-REVIEWED ARTICLES}\vspace{-8pt}
\begin{hangparas}{.25in}{1}
\textbf{Cheston, H.}, Cross, I., \& Harrison, P. M. C. (2024). Trade-offs in Coordination Strategies for Duet Jazz Performances Subject to Network Delay and Jitter. \emph{Music Perception}, \emph{42}(1), 48–72. \href{https://doi.org/10.1525/mp.2024.42.1.48}{https://doi.org/10.1525/mp.2024.42.1.48}

\textbf{Cheston, H.}, Schlichting, J. L., Cross, I., \& Harrison, P. M. C. (2024). Jazz Trio Database: Automated Annotation of Jazz Piano Trio Recordings Processed Using Audio Source Separation. \emph{Transactions of the International Society for Music Information Retrieval}, \emph{7}(1). \href{https://doi.org/10.5334/tismir.186}{https://doi.org/10.5334/tismir.186}

\textbf{Cheston, H.}, Schlichting, J. L., Cross, I., \& Harrison, P. M. C. (2024, Accepted Pending Minor Revisions). Rhythmic Qualities of Jazz Improvisation Predict Performer Identity and Style in Source-Separated Audio Recordings. \emph{Royal Society Open Science}
\end{hangparas}

\subsection{PREPRINTS}\vspace{-8pt}
\begin{hangparas}{.25in}{1}
\textbf{Cheston, H.}, Van Balen, J. \& Durand, S. (2024). Automatic Identification of Samples in Hip-Hop Music via Deep Metric Learning and an Artificial Dataset. \emph{arXiv}. 
\end{hangparas}

\subsection{CONFERENCE PROCEEDINGS}\vspace{-8pt}
\begin{hangparas}{.25in}{1}
\textbf{Cheston, H.}, Cross, I., \& Harrison, P. (2023). An Automated Pipeline for Characterizing Timing in Jazz Trios. Paper presented at the DMRN+18: Digital Music Research Network Workshop, Queen Mary University of London, UK. \href{https://www.qmul.ac.uk/dmrn/media/dmrn/DMRN-18-Proceedings.pdf}{[Proceedings]}

\textbf{Cheston, H.}, Cross, I., \& Harrison, P. (2023). Modelling Coordination Strategies in Improvised Musical Performances by Skilled Jazz Groups. Paper presented at the 16th International Conference of Students of Systematic Musicology (SysMus23), Sheffield University, UK. \href{https://drive.google.com/file/d/14F5Xe8qfxfWpMW6pGuIl8fcaIYbmFDtt/view}{[Proceedings]}

\textbf{Cheston, H.}, Cross, I., \& Harrison, P. (2023). Coordination Strategies in Networked Jazz Performances. Paper presented at the 17th International Conference on Music Perception and Cognition (ICMPC), Nihon University, Tokyo, Japan. \href{https://icmpc17.com/proceedings-in-zip/ICMPC17-APSCOM7-e-Proceedings.zip/}{[Proceedings]}

\textbf{Cheston, H.}, Cross, I., \& Harrison, P. (2022). The Effects of Variable Latency Timings and Jitter on Networked Musical Performances. Poster presented at the 15th International Conference of Students of Systematic Musicology (SysMus22), University of Ghent, Belgium. \href{http://hdl.handle.net/1854/LU-01GVD6WBCVPGRAHMNYR1CEXF57}{[Proceedings]}

\textbf{Cheston, H.}, Cross, I., \& Harrison, P. (2022). Measuring the Effects of Variable Latency Timings on Networked Jazz Performances. Poster presented at the SEMPRE 50th Anniversary Conference, Senate House, University of London, UK. \href{https://drive.google.com/file/d/1P72Orm1gqSI4_gOah3Ueb9fdCpriVY_M/view}{[Proceedings]}

\textbf{Cheston, H.} (2022). “Turning the Beat Around”: Time, Temporality, and Participation in the Jazz Solo Break. Paper presented at the 13th Conference on Interdisciplinary Musicology: 'Participation', University of Edinburgh, UK. \href{http://journals.ed.ac.uk/CIM22-Proceedings}{[Proceedings]}
\end{hangparas}

\subsection{OTHER PUBLICATIONS}\vspace{-8pt}
\begin{hangparas}{.25in}{1}
\textbf{Cheston, H.} (2022) Conference Review: Interdisciplinary Musicology 2022: 'Participation'. Royal Musical Association. \href{https://www.rma.ac.uk/2022/06/23/conference-review-interdisciplinary-musicology-2022-participation/}{[Conference review]}
\end{hangparas}


%----------------------------------------------------------------------------------------
%	TECHNICAL SKILLS
%----------------------------------------------------------------------------------------
\section{\centerline{TECHNICAL SKILLS}}\label{sec:skills}
\vspace{16pt} % Gap between title and text
\begin{itemize}[leftmargin=*]
	\setlength\itemsep{0pt}
    		\item[--]\textbf{Python} \faPython \hspace{2pt}: NumPy, Pandas, Matplotlib, Seaborn, PyTorch, Scikit-learn, SciPy, Statsmodels, OpenCV, Ray
    		\item[--]\textbf{JavaScript}: JQuery, DataTable, Plotly
    		\item[--]\textsf{R}: ggplot2, lme4, tidyverse
    		\item[--]\textbf{Misc.}: \faGit, \LaTeX, GCS, Docker, Stata, HTML, Apache Beam
\end{itemize}
    
%----------------------------------------------------------------------------------------
%	AWARDS AND PRIZES
%----------------------------------------------------------------------------------------
\section{\centerline{AWARDS AND PRIZES}}\label{sec:awards}
\vspace{8pt} % Gap between title and text
\begin{hangparas}{.25in}{1}
\textbf{Travel Award} (\emph{£200}), ICMPC \hfill  August, 2023\\
Travel Grant to present a paper at \emph{ICMPC17}, Tokyo, Japan

\textbf{Small Research Grant} (\emph{£500}), Royal Musical Association \hfill  August, 2023\\
Travel Grant to present a paper at \emph{ICMPC17}

\textbf{Travel Grant} (\emph{£1500}), Music \& Letters Trust \hfill  August, 2023\\
Travel Grant to present a paper at \emph{ICMPC17}

\textbf{Conference Paper Presentation Award} (\emph{£500}), Cambridge Digital Humanities \hfill  August, 2022\\
Travel Grant to present a paper at \emph{CIM}, Edinburgh, UK

\textbf{Project Incubation Award} (\emph{£2000}), Cambridge Digital Humanities \hfill  May, 2022\\
Awarded for continual development and testing of \emph{Audio-Visual Manipulator} software for studying musical performances in experiments \href{https://www.cdh.cam.ac.uk/research/cdh-funding/newmusicsoftwareplatform/}{[Project page]}

\textbf{Vice-Chancellor's Award} (\emph{£75,000}), Cambridge Trust \hfill  September, 2021 \\
Full scholarship (fees \& stipend) for Ph.D

\textbf{Musicology Prize} (\emph{£100}), University of Oxford \hfill  October, 2020\\
Awarded for highest mark in the 2020 MSt. cohort

\textbf{Louis Curran Graduate Scholarship} (\emph{£25,000}), University of Oxford \hfill  August, 2019\\
Full scholarship (fees \& stipend) for MSt. 

\textbf{Gibbs Prize} (\emph{£500}), University of Oxford \hfill  June, 2019 \\
Awarded for highest mark in the 2019 BA. cohort

\textbf{Clifford Smith Prize} (\emph{£130}), Christ Church, University of Oxford \hfill  June, 2019 \\
Awarded on basis of continued academic excellence

\textbf{Academic Scholarship} (\emph{£300} x2), Christ Church, University of Oxford \hfill  2017; 2018

\textbf{Collections Prize} (\emph{£50} x2), University of Oxford \hfill  September, 2018; September, 2017 \\
Awarded for performance in formative examinations
\end{hangparas}

%----------------------------------------------------------------------------------------
%	REFERENCES
%----------------------------------------------------------------------------------------
\section{\centerline{REFERENCES}}\label{sec:references}
\vspace{16pt}
\centerline{\textbf{Available on request}}

\end{resume} 
\end{document}